\documentclass{report}
\usepackage[utf8]{inputenc}
\usepackage[russian]{babel}

\begin{document}
\part{Механическое движение. 
Система отсчета. 
Траектория. Путь. 
Вектор перемещения и его проекции. 
Координатный и векторный способы описания движения. 
Закон движения. 
Скорость. 
Средняя скорость. 
Равномерное прямолинейное движение}

{\bf Механическое движение} —
изменение пространственного положения тела относительно других тел с течением времени. 

{\bf Траектория} —
линия, по которой двигалос тело.

{\bf Путь} —
длина участка траектории, пройденного материальной точкой за данный промежуток времени.

{\bf Перемещение} —
вектор, проведенный из начального положения материальной точки в конечное. 

{\bf система координат} —
набор осей, по которым исслудуется движение.

{\bf Материальная точка} —
тело, обладающее массой, размерами которого можно в данной задаче пренебречь.

{\bf Система отсчета} —
совокупность тела отсчета, связанной с ним системы координат и часов траектория — воображаемая линия, соединяющая положения материальной точки в ближайшие последовательные моменты времени.

{\bf Средняя скорость} —
скалярная величина, равная отношению пройденного пути к промежутку времени, в течение которого этот путь пройден. 
$$
v_{cp}=\frac{l}{t}
$$

{\bf Скорость} —
векторная физическая величина, равная пределу отношения перемещения тела к промежутку времени, в течение которого это перемещение произошло.
$$
\overrightarrow{v}=\lim_{\Delta t\rightarrow 0}\frac{\Delta \overrightarrow{r}}{\Delta \overrightarrow{t}}
$$

{\bf Закон Галилея. Относительное движение тел} —
скорость тела относительно неподвижной системы отсчета равняется векторной сумме скорости тела относительно подвижной системы отсчета и скорости неподвижной системы отсчета относительно подвижной.
$$
v=v_1+v_2
$$

{\bf Уравнение движения} —
зависимость координаты от времени
$$
x=x_0+S
$$
уравнение движения позволяет определяить положение тела в любой момент времени.

{\bf Равномерное прямолинейное движение} —
равномерным называется движение, при котором тело за любые равные промежутки времени проходит одинаковые пути.
$$
x=x_0+vt
$$

{\bf Физический смысл скорости движения} —
быстрота изменения координат.



\part{Неравномерное движение. 
Мгновенная скорость. 
Ускорение. 
Равноускоренное движение. 
Закон равноускоренного движения. 
Графики координаты и скорости при равноускоренном движении. 
Криволинейное движение. 
Скорость и ускорение при криволинейном движении.}
\end{document}