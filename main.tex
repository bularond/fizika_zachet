\documentclass{report}
\usepackage[utf8]{inputenc}
\usepackage[russian]{babel}
\begin{document}
\renewcommand{\partname}{Билет}
\part{Механическое движение. 
Система отсчета. 
Траектория. Путь. 
Вектор перемещения и его проекции. 
Координатный и векторный способы описания движения. 
Закон движения. 
Скорость. 
Средняя скорость. 
Равномерное прямолинейное движение}

{\bf Механическое движение} —
изменение пространственного положения тела относительно других тел с течением времени. 

{\bf Траектория} —
линия, по которой двигалось тело.

{\bf Путь} —
длина участка траектории, пройденного материальной точкой за данный промежуток времени.

{\bf Перемещение} —
вектор, проведенный из начального положения материальной точки в конечное. 

{\bf система координат} —
набор осей, по которым исслудуется движение.

{\bf Материальная точка} —
тело, обладающее массой, размерами которого можно в данной задаче пренебречь.

{\bf Система отсчета} —
совокупность тела отсчета, связанной с ним системы координат и часов траектория — воображаемая линия, соединяющая положения материальной точки в ближайшие последовательные моменты времени.

{\bf Средняя скорость} —
скалярная величина, равная отношению пройденного пути к промежутку времени, в течение которого этот путь пройден. 
$$
v_{cp}=\frac{l}{t}
$$

{\bf Скорость} —
векторная физическая величина, равная пределу отношения перемещения тела к промежутку времени, в течение которого это перемещение произошло.
Физический смысл: быстрота изменения координаты.
$$
\overrightarrow{v}=\lim_{\Delta t\rightarrow 0}\frac{\Delta \overrightarrow{r}}{\Delta \overrightarrow{t}}
$$

{\bf Закон Галилея. Относительное движение тел} —
скорость тела относительно неподвижной системы отсчета равняется векторной сумме скорости тела относительно подвижной системы отсчета и скорости неподвижной системы отсчета относительно подвижной.
$$
v=v_1+v_2
$$

{\bf Уравнение движения} —
зависимость координаты от времени
$$
x=x_0+S
$$
уравнение движения позволяет определяить положение тела в любой момент времени.

{\bf Равномерное прямолинейное движение} —
равномерным называется движение, при котором тело за любые равные промежутки времени проходит одинаковые пути.
$$
x=x_0+vt
$$

{\bf Физический смысл скорости движения} —
быстрота изменения координат.

\part{Неравномерное движение. 
Мгновенная скорость. 
Ускорение. 
Равноускоренное движение. 
Закон равноускоренного движения. 
Графики координаты и скорости при равноускоренном движении. 
Криволинейное движение. 
Скорость и ускорение при криволинейном движении.}

{\bf Мгновенная скорость} —
скорость тела в данный момент времени.

{\bf Ускорение} —
физическая фелечина, равная отношению изменения скорости к промежутку времени, за которое это изменение произошло.
Физический смысл: быстрота изменения скорости.
$$
a=\frac{\Delta v}{\Delta t}
$$

{\bf Равноускоренное движение} —
движение, при котором скорость изменяется на одинаковую велечину за равные отрезки времени.
$$
\overrightarrow{v}= \overrightarrow{v_0}+\overrightarrow{a} \Delta t
$$

\part{Центростремительное и тангенциальное ускорения.}

{\bf Центростремительное ускорение} —
ускорение, характеризующее быстроту изменения направления линейной скорости при движении точки по окружности.
Пермендекилярно вектору скорости.
$$
a=\frac{v^2}{R}
$$

{\bf Тангенциальное ускорение} —
усорение тела, сонаправленное вектору двидения.
$$
a=\frac{\Delta v}{\Delta t}
$$

\part{Движение тела, брошенного под углом к горизонту. 
Закон движения. 
Траектория движения и её уравнение. 
Дальность полета и максимальная высота подъема.
Центростремительное и тангенциальное ускорение. 
Движение по окружности. 
Угловая скорость и угловое ускорение. 
Связь между угловыми и линейными характеристиками движения. 
Период и частота.}

{\bf Движение тела, брошенного под углом к горизонту} —
тело брошенное под углом $\alpha$ и скоростью $v$ движется в пространстве под действием силы тяжести.
Горизонтальная состовляющая сткорости $v\cos{\alpha}$, вертикальная состовляющая
$v\sin{\alpha}+gt$

{\bf Траектория} —
линия, по которой двигалось тело.
$$
\{x,y,z\}=\{x_0+S_x, y_0+S_y, y_0+S_y\}
$$

{\bf Максимальная дальность полета} —
$$
L=\frac{v^2_0\sin{2\alpha}}{g}
$$

{\bf Максимальная высота полета} —
$$
H=\frac{v^2_0sin^2{\alpha}}{2g}
$$

{\bf Движение по окружности} —
криволинейное движение по окружности, из-за постоянной центростремительной скорости.

{\bf Период} —
время одного полного оборота.
$$
T=\frac{t}{N}
$$

{\bf Частота} —
количество оборотов за 1 секунду.

$$
\mathcal{V} = \frac{N}{t}
$$

{\bf Угловая скорость} —
отношение угла поворота ко времени поворота.
$$
\omega =\frac{\lambda}{\Delta t}
$$

{\bf Угловая скорость при равномерном движении} —
количество оборотов за $2\Pi$ секунд.
$$
\omega =\frac{2\pi}{T}=2\pi \mathcal{V}
$$

{\bf Скорость движения по окружности} —
$$
v=\frac{2\pi R}{T}=2\pi R \mathcal{V} = \omega R
$$

\part{Относительность механического движения. 
Формула сложения скоростей. 
Закон инерции Галилея. 
Первый закон Ньютона. 
Инерциальные системы отсчета. }

{\bf Закон Галилея. Относительное движение тел} —
скорость тела относительно неподвижной системы отсчета равняется векторной сумме скорости тела относительно подвижной системы отсчета и скорости неподвижной системы отсчета относительно подвижной.
$$
v=v_1+v_2
$$

{\bf Иннерция} —
явление сохранения телом скорости по направлению и значению.

{\bf Инертность} —
Свойство тела сохранять свою скорость по напрвлению и значению.

{\bf Масса} —
мера инертности.

{\bf Инерциальная система отсчета} —
система отсчета, тело отсчета которого движется равномено и прямолинейно или покоится.

{\bf Первый закон Ньютона} —
существую такие системы отсчета, относительно которых тела движутся равномерно и прямолинейно или 
покоятся, если на них не действует сила или действие сил компенсируется.

\part{ Масса. 
Сила. 
Второй закон Ньютона. 
Сложение сил. 
Измерение сил. 
Взаимодействие тел. 
Третий закон Ньютона. 
Принцип относительности Галилея.}

{\bf Масса} —
мера инертности.

{\bf сила} —
мера взаимодействия.

{\bf Второй закон Ньютона} —
ускорение тела прямо пропорционально равнодействующей всех сил, действующих на тело, и обратно
пропорионально массе этого тела
$$
\overrightarrow{a}=\frac{\overrightarrow{F}}{m}
$$

{\bf Третий закон Ньютона} —
при взаимодействии возникает две силы, равные по занчению друг-другу по велечине и противоположные
по направлению, приложенные к разным телам.

{\bf Равнодействующая всех сил} —
векторная сумма всех сил, действующих на тело.

\part{Сила упругости. 
Упругие и неупругие деформации. 
Закон Гука. 
Модуль Юнга. 
Движение под действием силы упругости. 
Силы трения. 
Сухое трение: трение покоя и скольжения. 
Коэффициент трения. 
Вязкое трение. 
Движение под действием силы трения. 
Движение и трение покоя. 
Тормозной путь. 
Время торможения.}

{\bf Дефформирование} —
изменение формы или объема тела. Дефформации бывают:

1. Упругие исчезают после перкращения действия деформирующей силы.

2. Пластические (не упругие) - чистично или полностью сохраняется после прекращения действия 
деформирующей силы.

{\bf Сила упругости. Закон гука} —
сила, возникающая в теле в результате его деформации и стремящаяся вернуть тело в исходное положение.
$$
F=kx
$$

{\bf } —

{\bf } —

{\bf } —

{\bf } —

{\bf } —

{\bf } —

\end{document}